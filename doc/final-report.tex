\documentclass[11pt, a4paper]{article}

\usepackage{amsmath}
\usepackage{amssymb}

% fonts
\usepackage{xeCJK}
\setCJKmainfont[BoldFont=SimHei]{SimSun}
\setCJKfamilyfont{hei}{SimHei}
\setCJKfamilyfont{kai}{KaiTi}
\setCJKfamilyfont{fang}{FangSong}
\newcommand{\hei}{\CJKfamily{hei}}
\newcommand{\kai}{\CJKfamily{kai}}
\newcommand{\fang}{\CJKfamily{fang}}

% style
\usepackage[top=2.54cm, bottom=2.54cm, left=3.18cm, right=3.18cm]{geometry}
\linespread{1.5}
\usepackage{indentfirst}
\parindent 2em
\punctstyle{quanjiao}
\renewcommand{\today}{\number\year 年 \number\month 月 \number\day 日}

% figures and tables
\usepackage{graphicx}
\usepackage[font={bf, footnotesize}, textfont=md]{caption}
\makeatletter
    \newcommand\fcaption{\def\@captype{figure}\caption}
    \newcommand\tcaption{\def\@captype{table}\caption}
\makeatother
\usepackage{booktabs}
\renewcommand\figurename{图}
\renewcommand\tablename{表}
\newcommand{\fref}[1]{\textbf{图\ref{#1}}}
\newcommand{\tref}[1]{\textbf{表\ref{#1}}}
\newcommand{\tabincell}[2]{\begin{tabular}{@{}#1@{}}#2\end{tabular}} % multiply lines in one grid

\usepackage{listings}
\lstset{basicstyle=\ttfamily}

\usepackage{xcolor}
\renewcommand{\r}{\color{red}}

\usepackage{url}

% start of document
\title{\textbf{\texttt{mydb}数据库设计报告}}
\author{\kai 朱俸民 \and \kai 温和}
\date{\kai \today}

\begin{document}

\maketitle

\section{功能介绍}

\texttt{mydb}是一个简单的面向单用户、单线程的关系数据库管理系统,支持增、查、删、改四项基本数据库查询功能,通过SQL的一个子集实现数据库与用户的交互。除了支持基本的增、查、删、改功能外,\texttt{mydb}具有如下扩展功能:

\begin{enumerate}
    \item 索引:使用\texttt{std::set},支持对任意列建立索引加速查询;
    \item 域完整性约束:支持\texttt{CHECK}语句的创建,并在更新数据库时进行约束检查;
    \item 外键约束:支持\texttt{FOREIGN KEY}的创建,并在更新数据库时进行约束检查;
    \item 模糊查询:支持使用\texttt{LIKE}语句进行模糊匹配;
    \item 支持多表连接查询,包括三个表以上的连接;
    \item 聚集查询:支持将\texttt{SUM},\texttt{AVG},\texttt{MAX},\texttt{MIN}这四个函数应用于表中的某些列;
    \item 分组聚集查询:支持查询时的\texttt{GROUP BY}语句,可以处理聚集查询与分组聚集查询并存的查询;
    \item 完整的建表属性限制:建表时除了允许\texttt{NOT NULL},\texttt{PRIMARY KEY},\texttt{FOREIGN KEY}与\texttt{CHECK}限制外,还可以设置\texttt{UNIQUE},\texttt{AUTO\_INCREMENT}与\texttt{DEFAULT}限制;
    \item 在查询时,支持\texttt{BETWEEN}和\texttt{IN}操作符;
    \item 远程连接:允许远程开启\texttt{mydb}服务端后,本地用\texttt{mydb}客户端进行连接,通过SQL语句交互的方式进行数据库操作;
    \item 采用MySQL风格打印查询结果,易于用户查看。
\end{enumerate}

\section{系统架构}



\section{模块设计}

\subsection{记录管理模块}

\subsection{系统管理模块}

\subsection{SQL解析模块}

\subsection{SQL执行模块}

\subsection{用户交互模块}

\section{测试结果}

\subsection{基本查询功能}

\subsection{扩展查询功能}

\subsection{远程连接功能}

\section{小组分工}

请见\tref{jobs}.

\begin{center}
    \tcaption{小组分工}\label{jobs}
    \begin{tabular}{ll}
        \toprule
        功能 & 负责人 \\
        \midrule
        记录管理 & 朱俸民 \\
        系统管理 & 温和 \\
        SQL解析与语义检查 & 朱俸民 \\
        查询执行 & 温和 \\
        交互终端与远程连接 & 朱俸民 \\
        索引 & 温和 \\
        \bottomrule
    \end{tabular}
\end{center}

\section{SQL语法支持}

详见\textit{SQL Support Document.pdf},或前往Github:\url{}。

\section{使用方法}

详见\textit{README.md},或前往Github:\url{https://github.com/paulzfm/mydb#mydb}。

\begin{thebibliography}{9}

\bibitem{bib1} Bison 2.3. \url{https://engineering.purdue.edu/~milind/ece573/2015fall/project/bison.html#Calc_002b_002b-_002d_002d_002d-C_002b_002b-Calculator}.

\bibitem{bib2} SQL Tutorial. \url{http://www.w3schools.com/sql/default.asp}.

\bibitem{bib3} 冯建华, 周立柱. 数据库系统设计与原理. 

\bibitem{bib4} cppreference. \url{http://en.cppreference.com/}.

\bibitem{bib5} RapidJSON. \url{http://rapidjson.org/index.html}.

\end{thebibliography}

\end{document}
